\documentclass[12 pt]{article}
\usepackage[margin=1in]{geometry}
\usepackage[colorlinks=true]{hyperref}
\title{CS291: Scientific Software Engineering}
\author{Nathan Collier}
\begin{document}
{\large 
  \begin{flushleft}
    CS291: Scientific Software Engineering

    Nathan Collier
    
    Fall 2012
    
    \rule{\linewidth}{0.25mm}
\end{flushleft}}

\paragraph{Details} Class meets Saturdays and Tuesdays from 4:00 to 5:30 pm in Building 9-3224. 

\paragraph{Prerequisites} Programming experience and familiarity with basic discrete and numerical algorithms.

\paragraph{KAUST course description}

Practical aspects of application development for high performance
computing. Programming language choice; compilers; compiler
usage. Build management using make and other tools. Library
development and usage. Portability and the GNU auto-configure
system. Correctness and performance debugging, performance
analysis. Group development practices and version control. Use of
third party libraries and software licensing.

\paragraph{Alternative course description}

The purpose of this class is to teach the principles of software
development with particular emphasis in the development of scientific
applications. Many computational scientists come from application
areas and develop software as best they can, unaware of many practices
which are common in computer science. These practices are mostly
learned `on the street' and infrequently taught in a formal class
setting. We will learn scientific software development actively by
learning to participate in a current project.

\paragraph{Reference Material}

There is no formal text for this class. There are a few texts
available in the KAUST library on software engineering in general. I
have also included some papers written about software in science.

\begin{itemize}
\item Textbooks on software engineering \cite{Sommerville2011,Nanz2011}
\item Opinions on the role of scientific software engineering in academia \cite{Todorov2012,Dirk2012}
\item Thoughts on scientific software \cite{Katzgraber2010,Hannay2009,Smith2007,Barnes2010}
\item Miscellaneous writings and tutorials \cite{make,Prentice1998,Gube2011,Spolsky2012}
\end{itemize}

\paragraph{Rough outline}

The following is a list of topics which we will cover in varying
amounts of depth and not necesarily in this order.
\begin{itemize}
\item Defining your problem: understanding the data and algorithm flow
\item Choosing a programming language: right tool for the job
\item Write the program: style and document as you go
\item Keep your source pure: version control
\item Keep your output pure: unit testing
\item Brief overview of compilers and how to use them
\item Make your life easier: makefiles
\item Debugging (gdb/valgrind/compiler options)
\item Making the code more efficient: profiling
\item Running on clusters and other high performance issues
\item Creating your own libraries
\item Using other libraries
\item Visualization tools
\end{itemize}

\paragraph{Grading}

The course will be mostly based on projects. I do not want to give
artificial practice assignments which are not useful to the
student. Thus the projects are designed to be immediately beneficial
to the student and/or the scientific community. Here I will give a few
samples:
\begin{itemize}
\item Starting with some code that you either wrote or perhaps use,
  cleanup the project by adding makefiles, compiler options, unit
  tests, documentation. Improve the code by making sure it conforms to
  a style. Check it thoroughly for memory problems. Profile the
  original version of the code and see what could be done to increase
  efficiency. Add version control to document your changes.
\item Search for software projects in your field and write a summary
  report which compares the projects. What languages are used? Is the
  software commerical or open source? How well documented? For what
  problem sizes is the code useful?
\item Identify an open source scientific project and a feature which
  you would like to add. Implement the feature, add documentation and
  testing, and submit the patch to the maintainers of the project.
\end{itemize}
In addition to the projects, I will periodically assign readings for
which I will give random quizzes. The final grade will be 80\%
projects and 20\% quizzes.

\clearpage
\bibliographystyle{alpha} \bibliography{course}
\end{document}
